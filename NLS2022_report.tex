\documentclass[a4paper,11pt]{article}
\usepackage[margin=3cm]{geometry} % Make more reasonable margins in document
\usepackage[parfill]{parskip}  % Remove LaTeX indents, and add interparagraph space
\usepackage[utf8]{inputenc}
\usepackage[english]{babel}

\usepackage{enumitem} % Enumerations met letters in de plaats van cijfers

%%%%%%%%%%%%
%%% MATH %%%
%%%%%%%%%%%%
\usepackage{mathtools}
\usepackage{amssymb}
\usepackage[output-decimal-marker={,}]{siunitx}
\newcommand\Log[1]{ \mathop{{}^{#1}\mathrm{log}} }
\DeclarePairedDelimiterX{\norm}[1]{\lVert}{\rVert}{#1}

% https://tex.stackexchange.com/questions/94032/fancy-tables-in-latex
\usepackage[table]{xcolor}
\usepackage{booktabs}

\usepackage[utf8]{inputenc}
\usepackage{pdfpages}

%Visuals
\usepackage{graphicx}
\usepackage{subcaption}
\usepackage[colorlinks,allcolors=violet]{hyperref}
\usepackage{url}
\usepackage[T1]{fontenc}
\usepackage{lmodern} 
\usepackage{algorithm}
\floatname{algorithm}{Algoritme}
\usepackage{algorithmic}
\usepackage{courier}
\usepackage{wrapfig}
\usepackage[export]{adjustbox}
\usepackage{wasysym}

%Figures
\usepackage{pgfplots}
\pgfplotsset{compat=newest}
%% the following commands are needed for some matlab2tikz features
\usetikzlibrary{plotmarks}
\usetikzlibrary{arrows.meta}
\usepgfplotslibrary{patchplots}
\usepackage{grffile}
\usepackage{amsmath}
\usepackage{caption}
\usepackage{float}
\usepackage{cleveref}
\usepackage{multirow}
\usepackage{titlesec}
\usepackage{listings}

\usepackage{multicol}
\usepackage{tikz}
\lstset{
	language=Matlab,
	%    basicstyle={\ttfamily \small},
	basicstyle={\ttfamily \small},
	%    keywordstyle=\underline,
	numberstyle={\footnotesize},
	%    morekeywords={ones,mod,isprime,inline,unique,factor,@},
	%    flexiblecolumns=false,
	%    emph={gamma,beta},
	%    emphstyle=,
	columns=fullflexible,
	%    columns=flexible,
	%    commentstyle={\slshape},
	%    commentstyle={\normalfont},
	commentstyle={\ttfamily},
	stringstyle={\ttfamily \bfseries},
	showstringspaces=false,
	%    indent=1em,
	%    xleftmargin=0.5em,
	breaklines=false,
	%    frame={l},
	captionpos={t},
	upquote=true, % such that we can copy-paste the code...
	%    mathescape=true,
	%    frame=single,     % boxed in a single line
	%    frame=L,          % double line on the left
	%    frame=l,          % single line on the left
}

\newcommand*\circled[1]{\tikz[baseline=(char.base)]{
            \node[shape=circle,draw,inner sep=0.1pt] (char) {#1};}}
\newcommand{\note}[1]{{\colorbox{yellow!40!white}{#1}}}
\titleformat{\section}
{\normalfont\Large\bfseries}{\boxed{\text{Assignment \thesection}}}{1em}{}

\setlength{\pdfpageheight}{11in}


\title{Nonlinear Systems \\[1ex]
    \Large \textsc{Assignments}}
\author{Elias Wils}
\date{\today}

\begin{document}

\maketitle
\newpage
\tableofcontents

\newpage
\section{Stability of equilibrium points and \\bifurcations}    
\subsection{A simple population model}
\paragraph{Question 1}\: The position and number of equilibrium points depends on the value of $\alpha$ and $\beta$.
In \Cref{tbeq1} the different possible cases are listed together with information about the equilibrium points.
When looking at the graph of $\dot{N}$ against $N$, a negative parabola can be observed, intersecting the $N$-axis in two points.
When solving the quadratic equation for $\dot{N}=0$ with $\alpha$ and $\beta$ as unknown parameters, expressions shown in \eqref{eqN} for
the equilibrium points ensue.
\begin{align}
	N_1 &= 0\\
	N_2 &= \frac{K(\alpha-\beta)}{\alpha}
	\label{eqN}
\end{align}
\begin{table}[H]
	\centering
	\begin{tabular}{|c|c|c|}
	\hline
	$\alpha<\beta$ & $\alpha>\beta$ & $\alpha=\beta$\\
	\hline
	$N_1$ is stable (\CIRCLE) & $N_1$ is unstable (\Circle) & $N_1=N_2$\\
	$N_2$ is unstable (\Circle) & $N_2$ is stable (\CIRCLE) & half stable equilibrium point (\RIGHTcircle)\\
	\hline
	\end{tabular}
	\caption{Characteristics of the different training algorithms for the given experiment, performed with and without noise. }
	\label{tbeq1}
\end{table}
The type of bifurcation that occurs here is called transcritical.\\\\
\paragraph{Question 2}\: For the practical example described here, the solution for $t\rightarrow\inf$ will converge to one 
of the equilibrium points. As $\alpha>\beta$, the second case in \Cref{tbeq1} is applicable. The only stable equilibrium point
is $N_2=\frac{K(\alpha-\beta)}{\alpha}=10\:470\:086$.\\

\subsection{Gene control model}
\paragraph{Question 1}\: When taking the repression rate $r=0$, only one fixed point is visible.


\end{document}